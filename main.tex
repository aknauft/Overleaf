\documentclass{article}
\usepackage[utf8]{inputenc}

\title{3.4 Rewrite: Rational Calculus}
\author{Andrew Knauft}
\date{November, 2020}

\usepackage{graphicx, amsmath, amsthm, amssymb}
\usepackage{hyperref}
\hypersetup{
    colorlinks=true,
    linkcolor=blue,
    filecolor=magenta,      
    urlcolor=blue,
}

\setlength{\parindent}{0cm}
\setlength{\parskip}{0.8em}

\begin{document}

\maketitle

\section{Background}
From the text: 
\begin{quote}
    How do you find a cubic polynomial \(f(x)\) with integer coefficients and rational roots, whose extrema and inflection points have rational coordinates?
\end{quote}
\section{Algorithm}
\begin{enumerate}
    \item Select two integers, \(r\) and \(s\), and use them to build the Eisenstein integer
    \[
        \alpha+\beta\omega = (1-\omega)(r+s\omega)^2
    \]
    \item Use the norm of \(\alpha+\beta\omega\) as the coefficient \(b\) on the cubic 
    \[
        f(x)=x^3-bx+c.
    \]
    This cubic has \(-\alpha\) and \(\beta\) as two of its roots.
    \item Solve for \(c\) by setting \(f(-\alpha)=0\), or \(f(\beta)=0\).
\end{enumerate}
Note that \(g(x) = f(x-h)\) is simply a horizontal translation of \(f(x)\), and as such will have integer features exactly when \(f(x)\) has integer features, provided \(h\) is an integer. This can be used to build a cubic where the quadratic coefficient is non-zero.

\section{Explanation}
See text, pages 124--125.

\section{Example}
\begin{enumerate}
    \item Let \(r=2,\ s=-1\). Then, 
    \begin{align*}
    \alpha+\beta\omega  &= (1-\omega)(2-\omega)^2 \\
                        &= (1-\omega)(4-4\omega+\omega^2)\\
                        &= (1-\omega)\left(4-4\omega+(-1-\omega)\right)\\
                        &= (1-\omega)(3-5\omega)\\
                        &= 3-5\omega -3\omega+5\omega^2 \\
                        &= 3-8\omega+5(-1-\omega) \\
                        &= -2-13\omega.
    \end{align*}
    \item Now, the norm of \(-2-13\omega\) is given by 
    \begin{align*}
        \alpha^2-\alpha\beta+\beta^2    &= (-2)^2-(-2)(-13)+(-13)^2 \\
                                        &= 4-26+169 \\
                                        &= 147
    \end{align*}
    Therefore, we take \fbox{\(-b = -147\)} as our linear coefficient.
    \item Solving for \(c\):
    \[
        f(-\alpha) = f(2) = (2)^3-147(2)+c = 0 \Rightarrow \fbox{c = 286}.
    \]
\end{enumerate}

Thus, we take our cubic to be \(f(x) = x^3-147x+286\), or any horizontal translation thereof.

\vspace{1.5in}
Desmos: \url{https://www.desmos.com/calculator/7j2exaonkk}

\end{document}
